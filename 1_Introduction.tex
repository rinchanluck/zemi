\begin{center}
  \Large
  松山研全体ゼミ勉強会 原子炉物理 座学
  \normalsize
\end{center}


\section{序論}
\subsection{原子炉/原子炉物理とは何か}
\begin{itembox}[l]{「原子炉物理」の定義}
  原子炉物理(Reactor Physics)とは、原子炉内で生じる\emph{中性子の挙動と原子核の反応を予測}する学問である。
\end{itembox}
\vskip\baselineskip
言葉の定義にあるように、原子炉の物理現象(Physics of Reactor)全てを取り扱うのではなく、核分裂反応を中核とする
中性子と原子核の相互作用に焦点をあてた学問である。しかしながら、中性子と原子核の相互作用の確率は原子炉の体系
に依存するため、他の物理現象の影響(例:温度、物質組成の変化)を受ける。したがって、最終的には様々な物理現象を考慮
する必要がある。
\begin{comment}
  BWRの核と熱水力の結合の例などを挙げる
\end{comment}

\vskip3\baselineskip
では、そもそも原子炉とは何であろうか。核燃料物質、核原料物質、原子炉及び放射線の定義に関する政令では
次のような定義を用いている。
\vskip\baselineskip
\begin{itembox}[l]{「原子炉」の定義}
  \emph{核燃料}を用い、\emph{制御可能な核分裂の連鎖反応}を\emph{中性子源無しに持続}できる装置、
  または持続する恐れのある装置以外のもの\\
  「核燃料物質、核原料物質、原子炉及び放射線の定義に関する政令」より要約
\end{itembox}
\vskip\baselineskip

核分裂反応は、重い原子核が2つに分裂し複数の中性子を放出する反応である。核分裂反応は主に中性子が原子核に吸収される
ことで発生するため、核分裂で発生した中性子(\emph{核分裂中性子})で核分裂反応を再び引き起こすことができる。
これが\emph{連鎖反応}と呼ばれる所以である。もし、核分裂中性子のみで一定の核分裂反応速度を維持できるなら、
中性子源無しに核分裂連鎖反応を持続させられる。これが原子炉の「\emph{臨界}」である。また、核分裂反応速度
が増加していく状態を\emph{超臨界}と呼ぶ。逆に核分裂反応が収束していく状態を\emph{未臨界}と呼ぶ。
原子炉には、体系の臨界性を変化させ核分裂連鎖反応を制御することが求められる。

\subsection{原子炉の利益}
原子炉には様々な用途が存在する。その用途は次のように大別できる。
\begin{itemize}
  \item 核反応による\emph{エネルギーの利用}
  \item 効率的な\emph{中性子源としての利用}(放射線利用)
\end{itemize}

1つ目は核分裂反応で発生したエネルギーを熱として取り出し利用することである。現在最も盛んなエネルギー利用法
は\emph{発電}である。原子力発電所では核エネルギーを用いて冷却材(主に水)を加熱し、蒸気タービンを回して発電する。
また、将来的にはより冷却材温度の高い高温ガス炉により、高熱を利用した\emph{水素製造}や\emph{製鉄}を行うことも期待されている。

現状主力となっている発電方法は、化石燃料の燃焼時の熱エネルギーを利用した火力発電であるが、温室効果ガス排出や
資源が地理的に偏在していること、燃料費が高く変動性もあることが課題となっている。また、水素製造や製鉄においても
化石燃料を用いることが課題となっている。例えば現在主流の水素製造法であるメタン水蒸気改質法の化学反応式は
\begin{equation}
  \ce{\textcolor{red}{\textbf{CH4}} + 2H2O -> \textcolor{red}{\textbf{CO2}} + 4H2}
\end{equation}
であり、天然ガス由来のメタンを消費し、二酸化炭素を排出することが欠点となっている。
製鉄については、高炉で鉄鉱石から銑鉄を得る際の正味の化学反応式は
\begin{equation}
  \ce{Fe2O3 + \textcolor{red}{\textbf{3CO}} -> 2Fe + \textcolor{red}{\textbf{3CO2}}}
\end{equation}
である。日本の場合、2023年度時点で二酸化炭素排出の1割強が製鉄由来であり、発電、道路輸送に次いで多い
二酸化炭素排出源となっている\cite{National-GHG-Inventory-Doc-Jpn-2025}。

原子炉をエネルギー源として利用すれば、これらの問題を解決することができる。
核分裂反応による温室効果ガス排出が無く、エネルギー密度が高いため少量の燃料を輸入するだけで
大量のエネルギー源を確保できる。天然ガスの主成分であるメタンの燃焼と、
主な核燃料であるウラン235の核分裂反応例の比較を次に示す。エネルギーの単位は
ウラン235、およびメタン分子1つあたりのエネルギー(メガ電子ボルト)、メタンの反応熱は
低位発熱量である。
\begin{align}
  \ce{^{235}U + n -> ^{92}Kr + ^{141}Ba + 3n} + \text{約} \textcolor{red}{\mathbf{200}} \  \text{[MeV/n]}\\
  \ce{CH4 + 2O2 -> CO2 + 2H2O} + \textcolor{red}{\mathbf{8.3 \times 10^{-6}}} \  \text{[MeV/n]}
\end{align}
発生するエネルギー核分裂反応はメタン燃焼の$2.4 \times 10^{7}$倍であり、エネルギー密度の差は歴然である。
これこそ原子炉のエネルギー利用の最大の利点である。

\vskip\baselineskip
\begin{shadebox}
  \toi 液化LNGガスと二酸化ウランペレットの\emph{質量に対する熱量}、\emph{体積当たりの熱量}を
  比較せよ。必要に応じて以下の値を使用すること。
  \begin{itemize}
    \item アボガドロ定数: $N_A = 6.022140857 \times 10^{23} \ \text{[mol$^{-1}$]}$
    \item 電子ボルトからジュールへの換算: $1.6021766208 \times 10^{-19} \ \text{[J/eV]}$
    \item 統一原子質量: $1.66053906892 \times 10^{-24} \ \text{[g/u]}$
    \item 二酸化ウラン(面心立方格子、蛍石型)\\
          単位格子あたりに酸素8つ、ウラン4つ 格子定数: $0.547 \ \text{[\si{\nm}]}$
    \item \ce{O2}原子量 $15.9994 \ \text{[u]}$
    \item \ce{^{235}U}、\ce{^{238}U}原子質量 それぞれ$ 235.0439 \ \text{[u]} $、$ 238.0508 \ \text{[u]} $
    \item \ce{^{235}U}核分裂あたりの平均エネルギー $ 202.5 \ \text{[\si{\MeV}]} $
    \item LNG発熱量 $ 54.6 \ \text{[\si{\mega\J/\kg}]} $
    \item LNG液密度 $ 0.460 \ \text{[\si{\kg/\L}]} $
  \end{itemize}
\end{shadebox}
\vskip\baselineskip

原子炉を用いれば、水素製造は高温の水蒸気\footnote{高温水蒸気電解法で\SI{700}{\degreeCelsius}、
熱化学法ISプロセスで\SI{900}{\degreeCelsius}}を用いて水から二酸化炭素排出無しで水素を製造できる。
正味の化学反応式は
\begin{equation}
  \ce{2H2O -> 2H2 + O2}
\end{equation}
である。この手法であれば、水以外の天然資源を使わず、かつ温室効果ガスを排出せずに
水素を製造できる。また、発生した水素を用いて水素還元製鉄と呼ばれる温室効果ガスを
排出しない製鉄も可能となる。
\begin{equation}
  \ce{Fe2O3 + 3H2 -> 2Fe + 3H2O}
\end{equation}
このように、電力以外の用途についても、原子炉は環境に良いエネルギー源として魅力的な存在である。

\vskip3\baselineskip

2つ目は、放射線源としての利用である。原子炉は中性子源無しに核分裂連鎖反応を維持できるため、加速器中性子源とは異なり
簡単かつ電力消費無しに大量の中性子を発生させることができる。中性子は核反応を用いて目的とする核種を製造したり、
減らしたい核種を他の核種に変換することができるため、原子炉中性子源は効率的な中性子源として利用されている。
主な核変換としては以下の例がある。
  \begin{itemize}
    \item 核燃料の増殖: \ce{ ^{238}U + n ->T[捕獲反応] ^{239}U ->T[$\beta^{-}$][23.45 min] ^{239}Np ->T[$\beta^{-}$][2.356 d] ^{239}Pu }
    \item 放射性同位体(Radioisotope; RI)製造: \ce{ ^{98}Mo + n -> ^{99}Mo ->T[$\beta^{-}$][2.749 d] ^{99m}Tc }
    \item 核廃棄物の核変換\\
          マイナーアクチノイド(\ce{U}、\ce{Pu}以外のアクチノイド): \[\ce{^{237}Np}\mbox{(半減期214万年)} \ce{ + n ->T[核分裂] ^{104}Tc + ^{130}Sn + 4n}\]
          \[\ce{^{104}Tc ->T[$\beta^{-}$][18.3 min] ^{104}Ru} \]
          \[\ce{^{130}Sn ->T[$\beta^{-}$][3.72 min] ^{130}Sb ->T[$\beta^{-}$][39.5 min] ^{130}Te} \mbox{(安定核)} \]
          長寿命核分裂生成物(LLFP:Long-lived fission products): \[\ce{^{129}I}\mbox{(半減期1570万年)} \ce{ + n -> ^{130}I ->T[$\beta^{-}$][12.36 h] ^{130}Xe} \mbox{(安定核)} \]
  \end{itemize} 
ウラン238から作られるプルトニウム239は、ウラン235と同じく核分裂を起こしやすい核種であり核燃料として利用できる。
この反応を使えば核燃料を燃やしながら核燃料物質を新たに製造でき、場合によっては消費量を超える量をも製造
できる可能性がある。このような燃料を増殖する原子炉を\emph{増殖炉}と呼ぶ。
核燃料以外の放射性同位体の製造としては、ここで挙げた医療用RIである\ce{^{99}Mo}/\ce{^{99m}Tc}を中心に
様々な核種の製造が期待されている。現在\ce{^{99}Mo}/\ce{^{99m}Tc}の製造は海外の原子炉のみであり、多くが
老朽化している上、半減期(半分の量が崩壊して他の核種になるのにかかる時間)が短いためテロや自然環境の
影響を受けやすい空輸で輸入せざるを得ない\cite{JRIA-MedRI-2021Nov}。
故に国内で製造できる原子炉を確保することが重要な課題となっている。
核廃棄物の核変換は、使用済み核燃料によって発生した廃棄物の長半減期の放射性同位体を
短半減期、安定核種に変換して処分の負担を軽減することが目的である。日本では、
放射性廃棄物を廃棄物・有用な物質に分離し、長寿命核種を変換する\emph{分離変換}が提案されており、
最終処分場の面積を最大$1/100$に抑え、1万年程度かかっていたウラン鉱石と同等の毒性に落ちるまでの
期間が300年まで短縮することができる。

核変換以外の放射線源としての用途としては、中性子ビームの利用がある。原子炉で発生した中性子を
ビームとして取り出し、物質の微細構造を分析したり、放射線透過による画像化、
ホウ素中性子捕獲療法によるがん治療を行うことができる。日本原子力研究開発機構の研究炉「JRR-3」は
中性子ビーム取り出し用の施設が備わっており、様々な研究に利用されている。



\subsection{核燃料}
主な核燃料物質としては\textbf{ウラン235}がある。ウランは鉱石や海水中に存在するため容易に
入手できる。しかしながら、ウラン235の天然存在比は0.72 at\%と少数であり、
それ以外のほぼ全ては核分裂を起こしにくいウラン238で占められている。故に天然存在比の
ウランでは原子炉を臨界にしにくい。故に、殆どの原子炉ではウラン235を\emph{濃縮}し
臨界しやすくして用いる。濃縮の度合いを\emph{濃縮度}と呼び、20 wt\%未満のものを\emph{低濃縮ウラン
(Low Enriched Uranium; LEU)}、それ以上のものを\emph{高濃縮ウラン(High Enriched Uranium; HEU)}
と呼ぶ。一方濃縮時に発生したウラン235が天然存在比未満のウランは減損ウランや\emph{劣化ウラン
(Depleted Uranium; DU)}と呼ばれる。一般的な商用炉の濃縮度は3~5 wt\%であり、LEUに分類される。
一方、研究炉や原子力艦船用原子炉\footnote{研究炉は小型炉心で大量の中性子を供給するために
高出力密度にする必要があることから、艦船用原子炉は船舶に搭載できるほど小型かつ長寿命にすることから、
それぞれHEUを必要とする}、核兵器\footnote{しばしば原子炉を制御できる理由は低濃縮度であるからという
説明がされるが誤りである。実際は原子炉が遅発臨界の領域で核反応を起こすからである。核兵器と一部の
特殊な研究炉のみ即発臨界を利用する。}などにはHEUを利用する。HEUは核兵器拡散リスクが高いため
商用炉では用いず、研究炉でも利用を最小限に抑える取り組みがなされている。なお、20\%を境界とするのは、
核兵器に必要な90\%付近までの濃縮に必要な分離作業の大半が20\%までの濃縮に費やされるためである。

近年では燃料交換の頻度を少なくするため、商用炉燃料の濃縮度を高くする潮流があり、
5~20 wt\%濃縮の高純度低濃縮ウラン(High-Assay Low Enriched Uranium; HALEU)
や5~10 wt\%濃縮の「LEU+」が次世代燃料として注目されている
\footnote{現状、国内生産の濃縮ウランは実質的に5\%の制約がある。これは濃縮度が5\%を超える場合、
「特定のウラン加工施設のための安全審査指針」の適用を受け、大幅な設備変更・投資が必要になって
しまうからである。なお、2025年現在で5\%を超える商用燃料は国外でも照射試験が開始されたばかり
である\cite{WNN-LEUplus-2025Apr}\cite{FEPC-LEUplus-2024Feb}。}。

他の核燃料としては\textbf{プルトニウム239}がある。この核種も核分裂を起こしやすく、
更に核分裂を起こしにくいウラン238から製造できるため、核燃料を燃やしながら増殖できる
\emph{増殖炉}を実現できる可能性がある。燃料の原料となるウラン238は、ウラン濃縮の過程で
廃品として発生する劣化ウランを用いれば良いので、ウラン資源の利用率を大幅に向上できるという
利点もある。日本が\emph{核燃料サイクル}の構築を目標としているのは、増殖炉を用いれば
資源の対外依存を抑えながら安定的にエネルギーを供給することができるからである。

% 章ごとの参考文献欄
\printbibliography[segment=\therefsegment,heading=subbibliography]

\newpage